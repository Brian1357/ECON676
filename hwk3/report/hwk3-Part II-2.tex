
The averages, standard errors and t-statistics of the coefficients are summarized in Table 9 below. The relationship between the portfolio beta and the average portfolio returns are displayed in Figure 3 below.

\begin{table}[ht]
\centering
\caption{summary statistics for the estimated $\gamma_0$ and $\gamma_M$}
\begin{tabular}{c | c | c}
		& $\gamma_0$	& $\gamma_M$	\\	[0.5ex] 	\hline
Coefficient	& 0.2769 		& 2.5748		\\
S. E.		& 0.0010		& 0.0028		\\
$t_{stat}$	& 282.22 		& 935.13		\\ 	[0.5ex]	\hline \hline
\end{tabular}
\end{table}

\begin{figure}[h]
	\centering
	\caption{ }
	\includegraphics[width = 0.9\columnwidth]{D:/works/duke_mqfe/econ676/problem_sets/ps3/figures/2d_2.png}
\end{figure}

When we use the tangent portfolio as the market proxy, the t-stat of $\gamma_M$ is significantly high, which indicates the market risk premium is statistically significant positive. In addition, the figure shows a linearly positive relationship between the portfolio returns and their betas. These results highlight the Roll(1977) critique that the linear relation between expected returns and beta cannot be used to test CAPM. This is because, when we use the full samples as the market proxy and use this market proxy to obtain the betas, the assets and the market proxy will always display a linear relation. The market proxy can be efficient when the true market portfolio is not, and the market proxy can be inefficient when the true market portfolio is. Therefore, we cannot reject or verify CAPM with the method given in this question, because we never know if the market is truly inefficient or the market proxy is wrong.